\documentclass[a4paper]{article}

\usepackage[utf8]{inputenc}
\usepackage{url}

\newcommand{\strong}[1]{{\normalfont\fontseries{b}\selectfont #1}}
\newcommand{\code}[1]{\mbox{\texttt{#1}}}
\newcommand{\pkg}[1]{\strong{#1}}
\newcommand{\proglang}[1]{\textsf{#1}}
\newcommand{\acronym}[1]{\textsc{#1}}

%% \VignetteIndexEntry{Introduction to the wordnet Package}

\usepackage{/home/Hornik/tmp/R/share/texmf/Sweave}
\begin{document}
\title{Introduction to the \pkg{wordnet} Package}
\author{Ingo Feinerer}
\maketitle
\sloppy

\begin{abstract}
  The \pkg{wordnet} package provides a \proglang{R} interface to the
  \proglang{WordNet} lexical database of English.
\end{abstract}

\section*{Introduction}
The \pkg{wordnet} package provides a \proglang{R} via \proglang{Java}
interface to the
\proglang{WordNet}\footnote{\url{http://wordnet.princeton.edu/}}
lexical database of English which is commonly used in linguistics and
text mining. Internally \pkg{wordnet} uses
\proglang{Jawbone}\footnote{\url{http://mfwallace.googlepages.com/jawbone.html}},
a \proglang{Java} \acronym{Api} to \proglang{WordNet}, to access the
database. Thus, this package needs both a working \proglang{Java}
installation, activated \proglang{Java} under \proglang{R} support, and a
working \proglang{WordNet} installation.

Since we simulate the behavior of \proglang{Jawbone}, its homepage
is a valuable source of information for background information and
details besides this vignette.

\section*{Loading the Package}
The package is loaded via
\begin{Schunk}
\begin{Sinput}
> library("wordnet")
\end{Sinput}
\end{Schunk}

\section*{Dictionary}
A so-called \emph{dictionary} is the main structure for accessing the
\proglang{WordNet} database. Before accessing the database the
dictionary must be initialized with the path to the directory where
the \proglang{WordNet} database has been installed (e.g.,
\code{/usr/local/WordNet-3.0/dict}). The function \code{initDict()}
searches environment variables (\code{WNHOME}) and default
installation locations such that the argument can be neglected in most
cases. It returns whether it found a valid installation and is ready
to access the database.
\begin{Schunk}
\begin{Sinput}
> validInstallation <- initDict()
> if (!validInstallation) stop("could not find WordNet installation")
\end{Sinput}
\end{Schunk}

After initialization we instantiate a dictionary object
\begin{Schunk}
\begin{Sinput}
> dict <- getDictInstance()
\end{Sinput}
\end{Schunk}
which we will use as our access point to the database.

\section*{Filters}
The package provides a set of filters in order to search for terms
according to certain criteria. Available filter types can be listed via
\begin{Schunk}
\begin{Sinput}
> getFilterTypes()
\end{Sinput}
\begin{Soutput}
[1] "ContainsFilter"   "EndsWithFilter"   "ExactMatchFilter"
[4] "RegexFilter"      "SoundFilter"      "StartsWithFilter"
[7] "WildcardFilter"  
\end{Soutput}
\end{Schunk}
A detailed description of available
filters gives the \proglang{Jawbone} homepage. E.g., we want to search
for words in the database which start with \code{car}. So we create
the desired filter with \code{getTermFilter()}, and access the first
five terms which are nouns via \code{getIndexTerms()}. So-called \emph{index
term}s hold information on the word itself and related meanings (i.e.,
so-called \emph{synset}s). The function \code{getLemma()} extracts the
word (so-called \emph{lemma} in \proglang{Jawbone} terminology).
\begin{Schunk}
\begin{Sinput}
> filter <- getTermFilter("StartsWithFilter", "car", TRUE)
> terms <- getIndexTerms(dict, "NOUN", 5, filter)
> sapply(terms, getLemma)
\end{Sinput}
\begin{Soutput}
[1] "car"          "car-ferry"    "car-mechanic" "car battery" 
[5] "car bomb"    
\end{Soutput}
\end{Schunk}

\section*{Synonyms}
A very common usage is to find synonyms for a given term. Therefore,
we provide the low-level function \code{getSynonyms()}. In this example
we ask the database for the synonyms of the term \code{company}.
\begin{Schunk}
\begin{Sinput}
> filter <- getTermFilter("ExactMatchFilter", "company", TRUE)
> terms <- getIndexTerms(dict, "NOUN", 1, filter)
> getSynonyms(terms[[1]])
\end{Sinput}
\begin{Soutput}
[1] "caller"         "companionship"  "company"        "fellowship"    
[5] "party"          "ship's company" "society"        "troupe"        
\end{Soutput}
\end{Schunk}
In addition there is the high-level function \code{synonyms()}
omitting special parameter settings.
\begin{Schunk}
\begin{Sinput}
> synonyms(dict, "company")
\end{Sinput}
\begin{Soutput}
[1] "caller"         "companionship"  "company"        "fellowship"    
[5] "party"          "ship's company" "society"        "troupe"        
\end{Soutput}
\end{Schunk}

\section*{Related Synsets}
Besides synonyms, synsets can provide information to related terms and
synsets. Following code example finds the antonyms (i.e., opposite
meaning) for the adjective \code{hot} in the database.
\begin{Schunk}
\begin{Sinput}
> filter <- getTermFilter("ExactMatchFilter", "hot", TRUE)
> terms <- getIndexTerms(dict, "ADJECTIVE", 1, filter)
> synsets <- getSynsets(terms[[1]])
> related <- getRelatedSynsets(synsets[[1]], "!")
> sapply(related, getWord)
\end{Sinput}
\begin{Soutput}
[1] "cold"
\end{Soutput}
\end{Schunk}

\end{document}
